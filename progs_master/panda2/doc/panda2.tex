     PANDA2 -- PROGRAM FOR MINIMUM WEIGHT DESIGN
               OF STIFFENED, COMPOSITE, LOCALLY BUCKLED PANELS

    David Bushnell, Dept. L1-24, Bldg. 204, (650) 424-3237


                       ABSTRACT

PANDA2 finds minimum weight designs of laminated composite flat or curved
cylindrical panels  or cylindrical shells with stiffeners in one or two
directions. Stiffeners can be blades, tees, angles, hats, or zees. Truss-core
sandwich panels can also be handled, as well as isogrid-stiffened panels
with added rings. The panels or shells may have initial imperfections of
the form of general, inter-ring, and local buckling modes. The panels or
shells can be loaded by as many as five combinations of in-plane loads,
edge moments, normal pressure, and temperature.  The material properties
can be temperature-dependent. The axial load can vary across the panel.
The presence of overall (bowing, ovalization, general buckling modal)
imperfections as well as inter-ring and local imperfections in the form of
the inter-ring and local buckling modes, respectively, can be included.
Constraints on the design include local buckling of stiffener segments and
rolling of stiffeners, local, inter-ring, and general buckling, maximum
displacement under pressure, maximum tensile or compressive stress along
the fibers and normal to the fibers in each lamina, and maximum in-plane
shear stress in each lamina. 

Local, inter-ring, and general buckling loads are calculated with use of
either closed-form expressions or with use of discretized models of panel
cross sections. The discretized model is based on one-dimensional
discretization similar to that used in the BOSOR4 computer code. An
analysis branch exists in which local post buckling of the panel skin is
accounted for. In this branch a constraint condition that prevents
stiffener popoff is introduced into the optimization calculations. The
postbuckling theory incorporated into PANDA2 is similar to that formulated
by Koiter for panels loaded into the far-postbuckling regime. 


PANDA2 can be run in five modes: 

 1. optimization,
 2. simple analysis of a fixed design,
 3. test simulation,
 4. design sensitivity, and
 5. load-interaction.

Plots of decision variables, design margins, and weight vs. design
iterations can be obtained following use of PANDA2 in the optimization
mode. Plots of user-selected behaviors and design margins vs load can be
obtained following use of PANDA2 in the test-simulation mode. Plots of
margins vs a user-selected design variable can be obtained following use
of PANDA2 in the design sensitivity mode. Plots of in-plane load
interaction curves and margins v. load combination number can be obtained
following use of PANDA2 in the load-interaction mode. 

There is a processor in the PANDA2 system called STAGSMODEL that
automatically generates an input file for the STAGS computer program.
Thus, STAGS, which is a general purpose nonlinear finite element analyzer,
can be used with reasonable ease to check the load-carrying capacity of
panels designed with PANDA2. 


                     INTRODUCTION

There is an extensive literature on the buckling and postbuckling behavior
of stiffened plates and shells. This literature covers metallic panels and
panels fabricated from laminated composite materials.  Leissa [1] has
gathered results from almost 400 sources on the buckling and postbuckling
behavior of flat and cylindrical panels made of composite material with
various stacking sequences and boundary conditions and subjected to
various in-plane loads. The emphasis in his survey is on theoretical
results, although some experimental results are included. He includes
several examples in which the effect of transverse shear deformation is
explored. Emphasis is given also to the effects of anisotropy on
bifurcation buckling and on postbuckling behavior. Wiggenraad [2] surveys
the literature on design of composite panels permitted to buckle locally
under operating loads.  Included in his survey are damage tolerance,
fatigue, and optimization. Arnold and Parekh [3] emphasize in their survey
and theoretical development the effect of in-plane shear load on the
postbuckling behavior of stiffened, composite cylindrical panels. Surveys
of earlier work on buckling of stiffened panels and shells appear in [4]
and [5]. 

Among the foremost contributors of information about buckling of stiffened
shells are Josef Singer and his colleagues at the Technion in Haifa,
Israel. In particular, the Baruch-Singer theory [6] for averaging the
properties of stiffeners over a shell surface while retaining the
important eccentricity effects has been incorporated into many widely used
computer programs for the stress, vibration, and buckling analysis of
stiffened shells. 

The literature in the field of buckling of stiffened shells can be divided
into three categories, one in which test results are emphasized, a second
in which structural analysis is emphasized, and a third in which optimum
designs are obtained. References [7] through [17] feature test results for
plates, shells, and stiffeners made of laminated composite material; [18]
through [25] feature structural analysis with structural properties fixed;
and [26] through [35] feature structural analysis with optimum
configurations sought in most cases via the widely used optimizers CONMIN
and ADS, written by Vanderplaats and his colleagues [36 - 38].  This is
just a sample of the literature on the subject. The reader is referred to
the surveys given in [1] through [5] and references cited there for other
sources. 


       REVIEW OF THE SCOPE AND PHILOSOPHY  OF PANDA2

The purpose of PANDA2 is to find the minimum weight design of a stiffened
flat or curved, perfect or imperfect, panel made of laminated composite
material. 


Geometry

The kinds of stiffening handled by PANDA2 are: 

 1. unstiffened
 2. T-shaped stiffeners
 3. J-shaped stiffeners
 4. Rectangular (blade) stiffeners
 5. Hat-shaped stiffeners
 6. truss-core sandwich (with no other
    stiffeners)
 7. isogrid-stiffened with added rings
 8. honeycomb core and foam core sandwich panels
 9. riveted Z-shaped stiffeners
10. substiffeners of rectangular cross section
    attached to the panel skin between major
    stringers and rings.

The properties of the panel are assumed to be uniform in the axial (x)
direction and periodic (consisting of repetitive modules) in the
circumferential (y) direction.  If the panel is other than of truss-core
sandwich construction or isogrid-stiffened, it may be stiffened by
uniformly spaced stringers alone, stiffened by uniformly spaced rings
alone, or stiffened by both rings and stringers. All stringers must be the
same. All rings must be the same. The rings can be different from the
stringers. Truss-core sandwich panels cannot be further stiffened by
either stringers or rings. In isogrid panels rings may be added. 

Note that the theory on which PANDA2 is based is valid only if the panel
is either unstiffened or, if stiffeners exist in either or both coordinate
directions, there are several of them within the span of the panel. One
cannot accurately determine the behavior of a panel with only one
stiffener, for example.  The panel, if axially stiffened, for example, has
a "field" of equally spaced, identical stringers. The same holds for a
ring-stiffened panel. 

In PANDA2 local buckling behavior is predicted from analysis of a single
module which is assumed to repeat several times over the width of the
panel. (With ring-stiffened panels the assumption is that the "skin"-ring
modules repeat several times over the length of the panel).
         ___                  ___                   ___
          |                    |                     |
          |                    |                     |
--------=====----------------=====-----------------=====---------
|<---Module No. 1-->|<-- Module No. 2 --->|<--- Module No. 3 -->|

A panel module consists of one stiffener plus skin of width equal to the
spacing between stiffeners. The single module is considered to be composed
of segments, each of which has its own laminated wall construction. 


      MODULE WITH T-SHAPED STIFFENER...

       Seg. No. 4-.
                    . !<------  w  ------>!
                     ._____________________
                                !   ^
           Segment No. 3 -----> !   !
                                !   !
               Seg. No. 2-.     !   h
                           .    !   !
         Seg. No. 1-.       .   !   !          .-Seg. No. 5
                   .         .  !   V         .(same as Seg. 1)
        --------------=====================--------------
                      !<------ b2 ------->!
        !<--- Module width  =  stiffener spacing, b --->!


    EXPLODED VIEW, SHOWING LAYERS and (SEGMENT, NODE) NUMBERS

                   Layer No. 1------.
                                     .
 (Segment,Node) = (4,1)_____________________(4,11)
                                       .
                              (3,11)     ._____ Layer No. j
                                !  
             Layer No. 1 -----> ! <----------- Layer No. k
                                !   
          Layer No. 1-.         !   
                       .        !   
    Layer No. 1-.       .       !               .-Layer No. 1
               .         .      !                .
              .           .   (3,1)               .
    --------------    =====================    --------------
  (1,1)   .   (1,11)(2,1)   . (2,6)    (2,11)(5,1)  .    (5,11)
           .                 .                       .
            .                 .                       .
             Layer No. m       Layer No. n    Layer No. m



General instability is predicted from a model in which the stiffeners are
"smeared" in the manner of Baruch and Singer [6] over the width
(stringers) and length (rings) of the panel. 


Boundary Conditions

In the PANDA2 system the panel is assumed to be simply supported along the
two edges normal to the plane of the screen (at y = 0 and at y = panel
width). The panel can be either simply supported or clamped along the
other two boundaries (at x = 0 and x = L), but the conditions must be the
same at both of these two boundaries.  The PANDA2 analysis is always
performed for simple support on all four edges. However, experience has
shown that for the purpose of calculating panel and general instability
load factors, clamping at x = 0 and at x = L can be simulated by the
analysis of a shorter simply supported panel: For example, an axially
compressed, flat panel clamped at x = 0 and x = L has general instability
loads approximately equal to those of a panel simply supported at x = 0
and x = L/sqrt[3.85]. In PANDA2, clamping at x = 0 and x = L is simulated
by calculation of general instability or wide column instability of a
simply supported panel with a shorter length, an "effective" length that
depends on the ratios of in-plane loads and on the "boundary layer length"
in the axial direction. This "effective" length is calculated by PANDA2
and is provided as output. 

In PANDA2 local buckling behavior and local stress concentrations near
stringers are assumed to be independent of the boundary conditions along
the four panel edges.  This is likely to be a good assumption if there are
more than two or three halfwaves in the local buckling pattern over the
length and width of the panel. 


Loading

PANDA2 allows the panel to be loaded by as many as five independent sets
of in-plane load combinations, Nx, Ny, Nxy, Mx, My, normal pressure p, and
temperature T(z) that is nonuniform over the panel cross section but
uniform in the (x,y) coordinates.  Buckling loads, postbuckling behavior,
and maximum stresses are calculated for each of the five load sets applied
by itself. PANDA2 determines the best design that is capable of surviving
all of the five load sets when each set is applied separately, as it would
be during different phases of a panel's lifetime or over different areas
of a large, uniform structure such as a complete cylindrical shell
subjected to spatially varying loads (See [35]). Associated with each of
the five independent load sets there can be two load subsets, Load Set A
and Load Set B. Load Set A consists of what are termed in the PANDA2
output as "eigenvalue loads": These are loads that are to be multiplied by
the critical buckling load factor (eigenvalue). Load Set B consists of
loads that are not multiplied by the critical buckling load factor. 


Types of Analysis

PANDA2 performs the following analyses:

1. CONSTITUTIVE LAW:

a. PANDA2 computes the integrated constitutive law [the 6x6 matrix C(i,j)
that relates reference surface strains, changes in curvature, and twist to
stress and moment resultants] for each segment of a panel module. 

b. It computes thermal resultants and strains from curing and from applied
temperature during service for each segment of a panel module. 

c. It computes the integrated constitutive law [the 6x6 matrix Cs(i,j)]
for the panel with either and both sets of stiffeners "smeared out".
("Smearing out" the stiffeners means averaging their properties over the
entire area of the panel as prescribed by Baruch and Singer [6]). 

d. It computes the tangent stiffness CTAN(i,j) of the panel skin in its
locally postbuckled state, if applicable. 

e. It computes the tangent stiffness CsTAN(i,j) of the panel with smeared
stiffeners, using CTAN(i,j) for the locally postbuckled stiffness of the
panel skin. 


2. EQUILIBRIUM:

a. PANDA2 computes bowing of the panel due to curing. 

b. It computes static response of the panel to uniform normal pressure,
using either linear or nonlinear theory.  Two problems are solved: 

b(1) overall static response of entire panel with smeared stiffeners, and 

b(2) local static response of a single panel module with a discretized
cross section. 

c. Average strain and resultant distribution in all of the panel module
cross section segments are determined for: 

c(1) the panel loaded by all loads except normal pressure. The effect of
bowing of the panel due to curing, applied temperature during service,
normal pressure, and edge moments is included, as well as the effect of
initial imperfections in the form of general, inter-ring, and local
buckling modes.

c(2) the panel loaded by normal pressure.

d. Stresses in material coordinates in each layer in each laminate
(segment) of the panel module are calculated either for the post-locally
buckled panel, or for the unbuckled panel, whichever is applicable. The
effect of a local imperfection in the form of the local buckling mode is
included, as well as axial bowing from either cure, temperature change
during service, pressure, edge moments, eccentrically applied axial load,
initial imperfection, or any combination of these effects. 

e. Tensile forces in parts of the stiffener web(s) that tend to pull the
web from the panel skin are calculated, and these forces are compared to a
maximum allowable "peel force" that the user has previously obtained from
peel tests on sample coupons that bear some similarity to the concept for
which he or she is seeking an optimum design. 


3. BUCKLING:

a. PANDA2 computes buckling load factors from a PANDA-type of analysis
(closed form, see Ref [4]) for general instability, local buckling of the
panel skin, local buckling of stiffener segments, and rolling of
stiffeners with and without participation of the panel skin. (IQUICK = 1
mode of operation). 

b. It computes the load factor for local skin buckling from a BOSOR4-type
[39] of analysis (finite strip method) in which the cross section of a
single panel module is discretized, as shown in Fig. 8. (IQUICK = 0 mode
of operation). 

c. It computes a load factor for wide column buckling from a BOSOR4-type
of analysis of a discretized single panel module. In this analysis the
reduced effective stiffness of the locally buckled panel skin is used, if
applicable. (IQUICK = 0 mode of operation). 

d. If the axial load varies over the width of the panel, PANDA2 computes a
load factor for general instability from a BOSOR4-type of analysis of the
entire panel with smeared stiffeners. The width of the panel is
discretized. Again, the reduced effective stiffness of the locally buckled
panel skin is used for this analysis, if applicable. 

e. A PANDA2 processor called PANEL generates a refined discretized model
of the entire panel width with stringer parts treated as flexible shell
branches. This model can be used directly as input to BOSOR4.

f. A PANDA2 processor called STAGSMODEL generates a STAGS finite element
model of as many skin-stringer modules of the panel as the user chooses.
STAGSMODEL produces data files, *.bin and *.inp, that can be used
directly as input to STAGS.

g. A PANDA2 processor called STAGSUNIT generates a STAGS finite element
model of a ring and stringer stiffened panel. STAGSUNIT produces input
data files, *.bin and *.inp, that can be used directly as input to STAGS.

h. PANDA2 computes various kinds of general, semi-general, and local
buckling loads from an alternative theory in which each type of buckling mode
is expanded in double trigonometric series of the buckling modal displacement
components, u, v, w.


Philosophy embodied in PANDA2

PANDA2 represents a more detailed treatment of certain behavior not
handled by PANDA [4].  In particular, optimum designs can be obtained for
imperfect panels, for panels with locally post-buckled skin for panels
with hat stiffeners, and for panels with truss-core sandwich construction
or isogrid-stiffened panels or panels with riveted Z-shaped stiffeners. In
addition, PANDA2 will handle linear or nonlinear static response to normal
pressure, panels with nonuniform axial loading, panels with edge moments,
and panels with thermal loading and temperature-dependent material
properties. Also, PANDA2 optimizes panels for multiple sets of loads,
whereas PANDA [4] optimizes for a single set of in-plane loads.  PANDA2
accounts for the effect of transverse shear deformation. 

Optimization is carried out based on several independently treated
structural models of the panel.  These might be classified into three
model types, as follows: 

1. Model type 1: Included are PANDA-type models [4] for general, local,
and panel buckling, bifurcation buckling of stiffener parts, and rolling
of stiffeners with and without participation of the panel skin.  With
PANDA-type models, buckling load factors are calculated from closed-form
equations rather than from discretized models. The formulas are given in
[4].  (See Table 1 and Figs. 1-4 of [4]). Model type 1 is used with both
the IQUICK = 1 and IQUICK = 0 modes of operation. 

2. Model type 2: Buckling load factors and post-local buckling behavior
are calculated for what is termed in PANDA2 a "skin-stringer panel
module."  A module includes the cross section of a stiffener plus the
panel skin of width equal to the spacing between stiffeners.  In this
model the panel module cross section is divided into segments, each of
which is discretized and analyzed via the finite difference energy method
[39] (finite strip method). Variation of deflection in the axial direction
is assumed to be harmonic [sin(nx) or cos(nx)].  This one-dimensional
discretization is similar to that used in the BOSOR4 and BOSOR5 programs
for the analysis of shells of revolution [39]. In fact, many of the
subroutines for buckling and vibration analysis are taken from BOSOR4 and
modified slightly in order to handle prismatic structures instead of
shells of revolution.  Model type 2 is used with the IQUICK = 0 mode of
operation. (NOTE: Model type 2 is not available for isogrid-stiffened
panels). 

The single module model gives a good approximation to the local skin
buckling mode if there are more than three equally spaced stringers in the
panel.  What goes on locally between interior stringers in a panel,
stringers which are rotating about their axes only, not bending, is only
weakly affected by the boundary conditions at panel edges that may be
several bays away. 

Both local and wide-column instability can be handled with the same
discretized structural model. For all except truss-core sandwich panels,
symmetry conditions are applied at the left and right edges of the single
module model, that is, symmetry conditions are applied midway between
stringers.  Edge conditions for the single module of the truss-core
sandwich panel are discussed in Ref. [43]. 

The wide column buckling model in PANDA2 is applied to an axial length of
panel between adjacent rings, or if there are no rings, to the entire
axial length of the panel, L or for clamped panels the modified length
L/sqrt{3.85} or other modified length derived in the PANDA2 mainprocessor.
The wide-column buckling load predicted from the single panel module is
always lower and, if the panel is not of the truss-core sandwich type and
is stringer-stiffened, usually reasonably close to the general instability
load of the entire width of the panel between rings because the axial
bending stiffness of a stringer-stiffened panel is usually much, much
greater than the transverse bending stiffness of the portion of the panel
between adjacent rings. Hence, the strain energy in the buckled panel, and
therefore the buckling behavior, is only weakly dependent on bending of
the panel transverse to the stringers. Therefore, the boundary conditions
along the edges of the panel parallel to the stringers are not important. 
On the other hand, local bending of the skin and local deformation of the
stringer parts in the wide column buckling mode may significantly affect
the wide column buckling load.  These effects are not included in the
closed-form PANDA-type model of general instability, but they are included
in the single panel module model of wide column buckling. 

The wide-column buckling model should not be used for prediction of
general buckling of truss-core sandwich panels or isogrid-stiffened
panels. It is too conservative because, unlike T, J, Blade, and
Hat-stiffened panels, sandwich panels and isogrid-stiffened panels have
bending stiffnesses in the x and y directions which are of the same
magnitude.  Therefore, one cannot ignore the longitudinal boundaries in
the calculation of general instability load factors of truss-core sandwich
panels.

In October, 1998 a new "skin"-ring discretized module model analogous
to the skin-stringer discretized module model was introduced. In this
model the stringers (if any) are smeared out, and a cross section
involving one ring and the skin-smeared-stringer panel wall of axial
length equal to one ring spacing (b(ring)/2 on each side of the ring
web root) is segmented and discretized exactly as described above for
the skin-stringer module. The "skin"-ring discretized module is generated
only for panels stiffened with rings that are blades or Tees or Jays or
Zees. (Not hats or truss-core).

3. Model type 3: Also included in the PANDA2 collection of models is a
discretized model of the entire width of the panel, treated in this case
with stiffeners smeared out. This model is introduced only if the axial
load varies across the width of the panel or if there exists normal
pressure. 

The purpose of PANDA2 is to yield optimum PRELIMINARY designs of rather
sophisticated panels that may experience very complex and very nonlinear
behavior. The goal is to do this without having to use large,
general-purpose programs with their elaborate data base management
systems.  This goal is achieved through the use of several separate
relatively simple models, each designed to capture a specific phenomenon,
rather than through the use of a single multi-dimensionally discretized
finite element model with a large number of degrees of freedom. 

For example, PANDA-type models (Model type 1) are used in the IQUICK = 1
mode of operation of the PANDA2 mainprocessor in order to obtain quick,
preliminary designs which one can then use as starting designs in
optimization analyses based on the more elaborate discretized panel module
model that is used in the IQUICK = 0 mode of operation. Also, PANDA-type
models are used to obtain buckling load factors in cases for which the
discretized panel module model is not applicable, to obtain knockdown
factors for the effect of in-plane shear loading, to obtain preliminary
estimates of how much growth in any initial panel bowing to expect under
compressive in-plane loads, and to check if it is likely that a curved
panel with uniform external pressure will collapse under the pressure
acting by itself. 

Models of type 2 (single discretized module) and type 3 (discretization of
entire width with smeared stiffeners) are used in tandem to obtain from
nonlinear theory the complex behavior of a stiffened plate or shell loaded
by normal pressure. Model type 3 is the only one that is valid if the
axial load varies across the width of the panel. 

In the panels designed by PANDA2 (except for isogrid-stiffened) the skin
between stringers and the stringer parts will deform if they are locally
imperfect, and even if they are perfect they may buckle well before
failure of the panel. The maximum stress components and therefore stress
constraints in the optimization analysis are computed including local
prebuckling deformation and local post buckling growth and modification of
the local skin buckling mode as predicted by a modified form of a theory
formulated by Koiter in 1946 [40, 35]. Model type 2 (single discretized
module) is the only model in PANDA2 valid for these analyses. 

After the optimum design is obtained, the user can, if no in-plane shear
load is applied, check the accuracy of the general instability load
predicted from the single-module model by running a multi-module model
with BOSOR4 [39].  The input data file for this multi-module model is
generated automatically by the "PANEL" processor of the PANDA2 system. 

A PANDA2 processor called "STAGSMODEL" has been written.  This processor,
using the PANDA2 data base for a panel optimized by PANDA2, generates
input files for the STAGS computer program and the STAGS postprocessor
[21, 22, 48].  With STAGSMODEL the user can check designs obtained from
PANDA2 with a widely used, general purpose, nonlinear static and dynamic
finite element program. (As of this writing STAGSMODEL will not handle
truss-core stiffened panels or isogrid-stiffened panels). 


Architecture of the PANDA2 system

As with PANDA [4], the program PANDA2 [35] consists of several
independently executable processors which share a common data base. In the
processor BEGIN the user supplies a starting design (perhaps a design
produced by PANDA).  In DECIDE the user chooses decision variables for the
optimization analysis and their upper and lower bounds, linking variables
and their factors of proportionality, and "escape" variables (explained in
[35]). In MAINSETUP the user chooses up to five sets of combined in-plane
loads and normal pressure; factors of safety for general instability,
panel (between rings) instability, local instability, and material
failure; strategy parameters such as number and range of axial half-waves
in the local buckling mode; and number of design iterations in the
optimization problem. The command PANDAOPT initiates a batch run of the
PANDA2 mainprocessor, which consists of two main branches:  in one branch
the structural analyses (stress, buckling and post-buckling) are performed
and in the other new designs are produced by the optimizer ADS [37, 38]. 
Results can be plotted with use of the processors CHOOSEPLOT and DIPLOT.
Input data files for BOSOR4 and STAGS can be generated with use of the
processors PANEL and STAGSMODEL, respectively. Searches for a global
optimum design can be conducted with use of the processor SUPEROPT, which
represents a runstream consisting of many PANDAOPTs with intermittant
establishment of new "starting" designs via a random process coded in
a processor called AUTOCHANGE.


Improvements to PANDA2

Details about improvements to PANDA2 since publication of the original
paper in 1987 are provided in [41, 42]. A summary of the most significant
modifications and additions is provided here. 

1. Plots of dimensions, objective, and design margins vs. design
iterations are generated by new processors, CHOOSEPLOT and DIPLOT,
described in [41, 42]. Many of the plots in [43] were generated
with the CHOOSEPLOT/DIPLOT processors. 

2. There is new flexibility with regard to the in-plane movability of the
edges as the panel is loaded by normal pressure. Indeed, with movable
boundaries of flat panels especially, the increased bowing and lack of
development of average in-plane tension caused by bending can greatly
reduce local buckling load factors, affecting the local postbuckling
behavior and therefore the stress constraints that influence the evolution
of the optimum design during optimization iterations. 

3. New logic has been introduced to generate a "knockdown" factor to
compensate for the inherent unconservativeness of smearing stiffeners in
the prediction of general (overall) and panel (between rings) instability,
especially in cases where there is significant applied in-plane shear
loading. The amount of knockdown is related (a) to the ratio of the
wavelength of the buckles in the in-plane coordinate y normal to the
stiffeners to the spacing b of the stiffeners, and (b) to the ratio of the
applied in-plane shear load to the axial load. 

4. Edge moments may be applied to the panel.

5. Temperature distributions may be applied to the panel and the material
may be temperature-dependent. (Material must still behave linearly,
however). 

6. If the panel is loaded by normal pressure, buckling and stress
constraints are generated for conditions both at the midlength and at the
ends of the panel.  This modification is important because the very
different distributions of axial compression over the skin-stringer cross
section cause very different buckling and stress behavior at these
different axial locations.  Behavioral constraint conditions corresponding
to both panel midlength and panel ends may influence the evolution of the
design during optimization iterations. 

7. A new behavioral constraint condition based on bending-torsional
buckling with a long axial wavelength has been introduced. 

8. There may now be different boundary conditions for the prebuckling and
bifurcation buckling phases of the problem.  This modification is
important for finding optimum designs of panels under normal pressure in
which the panel being optimized spans large rings. In the prebuckling
phase the rings act like clamps, preventing edge rotation because of
symmetrical behavior on either side of the ring.  However, practical
rings, especially those with open cross sections, are likely to be too
weak to prevent edge rotation in the bifurcation buckling phase of
behavior.  Therefore, simple support conditions are usually called for
then. 

9. If the panel is considered to be clamped in the prebuckling phase and
if it is axially stiffened, its overall response to normal pressure is
predicted from wide-beam bending theory in which the effect of axial
compression is included. The theory is taken from Roark [44]. At the
midlength of the panel the change in curvature is given by: 

        Kappax(midlength) = p*(a/sin(a) - 1)/[EI(eff)*k**2]    (1)

in which
           a = k*L/2 ;     k = SQRT[-Nx(eff)/EI(eff)]          (2)

where EI(eff) is the effective axial bending stiffness of the panel with
smeared stringers and Nx(eff) is the effective axial resultant (positive
for tension) acting on the panel. 

The effective axial bending stiffness is obtained through a PANDA-type [4]
general instability analysis of a wide panel clamped at its axially loaded
ends with the effective axial stiffness of the panel skin and base under
the stringer set equal to one half their nominal values in order to
account for the diminished effective flange width of a beam with a wide
flange. The quantity EI(eff) is calculated with use of the Euler buckling
formula for a clamped-clamped column,  P(Euler) = 4*pi**2*EI/L**2. Since
P(Euler) = LAMBDA*ABS(Nx), the effective axial bending stiffness EI(eff)
of the wide column is given by 

          EI(eff) = LAMBDA*ABS(Nx)*L**2/4*pi**2.               (3)

in which LAMBDA is the eigenvalue found from the PANDA-type buckling
analysis of the panel with smeared stiffeners. 

The effective axial resultant, Nx(eff) = EBEAMR*Nx, accounts for the
presence of in-plane shear Nxy combined with the axial load Nx. The factor
EBEAMR is given by EBEAMR = LAMBDA/LAMBDA2, in which LAMBDA is the general
instability buckling load factor with just Nx acting on the panel and
LAMBDA2 is the general instability buckling load factor with both Nx and
Nxy acting on the panel. 

The normal deflection at the midlength of the panel, Wpressure, is given
by: 

        w(midlength) = p*L*[tan(a/2) - a/2]/[2*k*Nx(eff)]      (4)

At the ends of the panel the change in curvature is given by:

         Kappax(ends) = -p*[1 - a/tan(a)]/[EI(eff)*k**2]       (5)  
 
For a panel with no axial load and with the panel skin considered to be
fully effective as the beam-column bends, Equations (1), (4) and (5)
assume the well-known forms: 

     Kappax(midlength) =  (1/24)*p*L**2/C44(neutral axis)      (6)
     w(midlength)      = (1/384)*p*L**4/C44(neutral axis)      (7)
     Kappax(ends)      = -(1/12)*p*L**2/C44(neutral axis)      (8)

The maximum web shear resultant, Nxy(due to pressure) is

         Nxy(p) = (1/2)*p*L*b/h                                (9)

In the formulas above, p is the normal pressure, positive as shown in Fig.
8, page 490 of [35], L is the length of the panel between the large rings
(which are not present as actual structures in the PANDA2 model, and
C44(neutral axis) is the bending stiffness per transverse arc length when
the reference surface is the neutral plane in the x-direction. 
 
The beam bending model is used only for the deformation of the entire
panel with smeared stiffeners. The local model for bending and stretching
under pressure (single module) has not been changed, except that the user
now has a choice as to whether the longitudinal edges are in-plane movable
or not. 


10. More general linking expressions and inequality constraints based on
the panel cross section dimensions have been introduced into the DECIDE
processor. A general expression for a linked variable now has the form: 
   
       (linked variable)  =  C1*(decision variable no. j1)
                            +C2*(decision variable no. j2)
                            +C3*(decision variable no. j3)
                            +etc (up to max. of 5 terms)
                            +C0                                   (10)
 
in which C1, C2,..;  and C0 are constants.  Inequality relations among
variables may have either of the two forms: 

       1.0 < f(v1,v2,v3,...)    or   1.0 > f(v1,v2,v3,...)        (11)
 
where the expression f(v1,v2,v3,...) has the form:
 
    f(v1,v2,v3,...) = C0 +C1*v1**D1 +C2*v2**D2 +C3*v3**D3 + ...
                             +etc (up to max. of 5 terms).        (12)
 
The variables, v1, v2, v3,..., can be any of the variables that are
decision variables or potential candidates for decision variables or
linked variables. 
 
11. Whereas formerly there were two choices for type of analysis,

   (1) optimization and
   (2) analysis of a fixed design at a fixed load,

there are now also third, fourth, and fifth choices:

   (3) test simulation
   (4) design sensitivity
   (5) in-plane load interaction

In the "test simulation" mode the user supplies starting loads, load
increments and the number of load steps.  PANDA2 calculates the response
of a panel of fixed design for a number of load steps until general
instability is detected or until the maximum number of load steps
specified by the user is reached. 

In the "design sensitivity" mode the user chooses a particular design
variable and beginning and ending values of that variable.  PANDA2
calculates the response with fixed loads and a number of panel designs in
which the user-chosen design variable increases from its beginning value
to its ending value.  Margins can be plotted v. the chosen design variable
with use of the PANDA2 processor "CHOOSEPLOT".. 

In the "in-plane load interaction" mode the user selects the type of
load-interaction, (Nx,Ny) with constant Nxy or (Nx,Nxy) with constant Ny
or (Ny,Nxy) with constant Nx, the ranges to be covered for the varying
in-plane loads in the interaction, and the number of load combinations for
which to obtain margins. PANDA2 yields interaction plots for user-selected
margins. 

12. The assumed displacement field for calculation of PANDA-type buckling
load factors has been modified for flat panels.  In [4b] a Rayleigh-Ritz
method for obtaining bifurcation buckling loads of anisotropic flat and
cylindrical panels is described. The assumed displacement field (u,v,w)
for the buckling mode is given by Eqs. (50), p 552, of [4b]: 

 u = A[n2**2*m1*sin(n1*y -m1*x) +n1**2*m2*sin(n2*y +m2*x)]
 v = B[      n2*sin(n1*y -m1*x) -      n1*sin(n2*y +m2*x)]     (13)
 w = C[         cos(n1*y -m1*x) -         cos(n2*y +m2*x)]

in which n1, n2, m1, m2 are given by

     n1 = n+mc;   n2 = n-mc;   m1 = m+nd;   m2 = m-nd          (14)

where c and d are the slopes of the buckling nodal lines as shown in Fig.
9 of [4b]. The axial and circumferential wave indices m and n are defined
as 

           m = M*pi/x(max);    n = N*pi/y(max)                 (15)

where M and N are the number of halfwaves over the axial length x(max) and
the circumferential length y(max), respectively. 

This formulation has been retained because it usually gives the best
results for curved (cylindrical) panels, especially if the curved panel
spans more than one radian.  However, while comparing buckling loads for
flat panels with unbalanced laminates from results obtained with the STAGS
computer program [21,22], especially when in-plane shear predominates, it
was found that the following assumed displacement pattern yields better
predictions: 

     u = A[m1*sin(n1*y -m1*x) + m2*sin(n2*y +m2*x)]
     v = B[n1*sin(n1*y -m1*x) - n2*sin(n2*y +m2*x)]            (16)
     w = C[   cos(n1*y -m1*x) -    cos(n2*y +m2*x)]

13. A rather elaborate strategy has been introduced to obtain improved
accuracy of buckling load factors for curved panels in which in-plane
shear is a significant load component.  Details of the strategy are
included in the file called "PANDA2.NEWS" [42], which forms part of the
PANDA2 literature distributed with the program. 

14. The local postbuckling theory, based on the work of Koiter [40], has
been improved to allow for change in the axial wavelength of the local
buckles as the panel is loaded further and further into the postbuckling
regime. 

15. Axial bending induced by neutral plane shift after local buckling has
been introduced into the local postbuckling anslysis. 

16. The PANDA2 user may now obtain plots of extreme fiber strains vs. load
(test simulation mode) at several user-selected points in the panel module
cross section. 

17. The move limits for decision variables during optimization cycles now
depend to a certain extent on the value of the gradients of the behavioral
constraint conditions. 

18. The thermal buckling capability of PANDA2 has been improved. It is now
possible to perform an ITYPE=3 analysis (test simulation) with thermal
loading only. 

19. A PANDA2 processor called "STAGSMODEL" has been written.  This
processor, using the PANDA2 data base for a panel optimized by PANDA2,
generates input files for the STAGS computer program and the STAGS
postprocessor [21, 22].  With STAGSMODEL the user can check designs
obtained from PANDA2 with a widely used, general purpose, nonlinear static
and dynamic finite element program. 

20. Often when a stiffened panel is designed to withstand loads much
higher than the local buckling load of the panel skin, there is "secondary
local buckling" or "mode jumping".  That is, in tests of optimized panels
bifurcation buckling of the already locally buckled panel skin has been
observed at loads considerably above the initial local buckling load. If a
panel is made of composite material which can delaminate relatively easily
this secondary buckling, which usually occurs with a loud snapping noise,
might well lead to early failure of the structure.  Therefore, it is
advantageous to be able to find optimum designs for which secondary
buckling of the panel skin is unlikely to occur. Coding has been
introduced into PANDA2 by means of which optimum designs of panels can be
obtained for which secondary buckling is postponed to loads that are
higher than the design load. 

21. PANDA2 now runs with UNIX-based operating systems.

22. Improvements have been made in the special case of a ring stiffened
panel or shell under uniform pressure. The "hungry horse" prebuckling
bending is now included in the analysis. See ITEM 116 of PANDA2.NEWS. 

23. Linear variation of the axial resultant across the height of stiffener
webs is now included in the calculation of local web buckling and local
rolling of stiffeners. 

24. Panels with isogrid stiffening with added rings can be handled with
the IQUICK = 1 (PANDA-type) models for buckling and stress. 

25. Predictions of the buckling and stress behavior of imperfect
unstiffened and stiffened cylindrical panels has been improved. The effect
on buckling load factors and stress for cylindrical panels with buckling
modal initial imperfections for local, panel (inter-ring), and general
instability modes, as well as overall ovalization, has been incorporated
into PANDA2. (See PANDA2.NEWS ITEM 124 and Ref. 4 under "Recent PANDA2
references" ). 

26. Predictions of buckling by PANDA-type (closed-form) buckling formulas
are now available in PANDA2 based on either Donnell equations or Sanders
equations.  A new input datum called ISAND is required for MAINSETUP and
PANDAOPT.  Details appear in ITEM 128 of PANDA2.NEWS. 

27. There is now a processor called AUTOCHANGE by means of which the
decision variables are automatically changed as described in ITEM 151 in
PANDA2.NEWS. This processor makes the task of looking for a global optimum
design much easier. AUTOCHANGE can be used by itself (via the command
"autochange"). AUTOCHANGE is run as part of the SUPEROPT process. SUPEROPT
looks for a global optimum design. See ITEM 152 in PANDA2.NEWS. 

28. PANDA2 ordinarily optimizes for conditions both at the midlength and
at the ends of a panel or cylindrical shell. Sometimes there may be high
stress concentrations within a relatively narrow "boundary length" of the
ends of a clamped panel or near heavy rings in a ring-stiffened
cylindrical panel or shell. It may be best to design a panel with
different properties in the end regions and more than a "boundary layer"
away from the ends. This can now be done. Judgment is required if the user
wishes to take advantage of this feature. Details and an example are
presented in ITEMs 175 and 378 of PANDA2.NEWS .

30. The section of PANDA2 having to do with buckling of imperfect
cylindrical panels and shells (see Paragraph No. 25 in this section) has
been considerably improved by incorporation of Johann Arbocz' extended
version of Koiter's Special (1963) theory for buckling of axisymmetrically
imperfect cylindrical shells. See panda2.news ITEM 201 for details. 

31. The permitted cross-section geometry of a truss-core sandwich panel
has been expanded beyond that shown in Ref. [43]: There may now be a
region b2 of finite width over which the truss core is integrated into
the panel face sheets, as shown below:

      Truss-core sandwich wall construction
       with extra segments of width b2....

                 < b2 >  ---- Seg. 3 ---->
      -----------======--------------------======-----------<-Upper skin
                /-Seg6>\S                 /      \        ^    middle surface
               /        \e              4/        \       |
              /        . \g.           g/          \      |
             /         |\ \           e/ /          \     h=(skin-middle-surf.
            /            \ \2        S/ /            \    |   to-skin-middle-
           /                \        /|/              \   |   surface height)
          /  --- Seg. 1----> \<Seg5-/ '                \  V
      ====--------------------======--------------------==--<-lower skin
                              < b2 >                         middle surface
          <---------- b ----------->

          -----> "y" coordinate direction

       A single module consists of Seg. 1 through Seg. 6.
       Seg. 4 has the same wall construction as Seg. 2.
       Soon you will be asked:
         "Are the segs. of width b2 thicker than face sheets?"
       If you answer N to this question, then:
          Seg. 5 has the same wall construction as Seg. 1.
          Seg. 6 has the same wall construction as Seg. 3.
       If you answer Y to this question, then:
          Seg. 5 has stacked wall construction: Seg.5 = Seg.1+Seg.2.
          Seg. 6 has stacked wall construction: Seg.6 = Seg.3+Seg.2.

See ITEMs 222 and 265 in PANDA2.NEWS for more details.


32. The prompting coding (PROMPTER library) was modified to permit the
user to supply default values of input simply by hitting "return". This
modification is especially important in MAINSETUP, where many of the
strategy input values require sophisticated knowledge on the part of
the user.  The following notice was added to the SAMPLE.CAS file and
is repeated twice in PANDA2.NEWS Items 171 and 224:

 ************ NOTE ************** NOTE ************** NOTE **************
 ************ NOTE ************** NOTE ************** NOTE **************
 ************ NOTE ************** NOTE ************** NOTE **************
 Some of the prompts for input require sophisticated knowledge on the
 part of the user. However, default options are available for most of
 the input. If you are not sure of your reply, simply hit "return" and
 the default value will automatically be supplied. If, upon your hitting
 "return", PANDA2 replies, "PLEASE SAY SOMETHING", then you must supply
 an input datum. If you are not sure what to supply, type "H" (for "Help")
 and more information will be provided.
 ************ NOTE ************** NOTE ************** NOTE **************
 ************ NOTE ************** NOTE ************** NOTE **************
 ************ NOTE ************** NOTE ************** NOTE **************


33. PANDA2 can now be used to find optimum cross sections of beams. The
user sets the total width, CIRC, of the panel equal to a single skin-stringer
module, b. Then PANDA2 asks the user if he/she wants to analyze a beam.
If the user answers Y, then PANDA2 resets the boundary conditions along
the two longitudinal edges of the one-module "panel" to free-free. Please
see ITEM 246 in PANDA2.NEWS for more details.


34. PANDA2 will now handle the following failure modes and phenomena in
sandwich panels or in panels in which one or more segments of the panel
module has a sandwich wall construction: face wrinkling, face dimpling,
core shear crimping, core crushing, core tension failure, core failure due
to excess transverse shear stresses applied to either the x-z or y-z faces
of the core, face sheet pull-off, the effects of initial face sheet
waviness. Also, PANDA2 will handle buckling of panels with segments on
elastic foundations. Please see ITEM 271 in the file 

         ...panda2/doc/panda2.news

for more details. 

35. PANDA2 now permits taking advantage of "bending overshoot" in
one-layered panel module segments. This means that different stress
allowables may be applied to the membrane and bending components of
the stress. Different "bending overshoot" constants, fxt, fxc, fyt,
fyc, and fxy may be applied for tensile ("t") and compressive ("c")
membrane stresses in the axial ("x") and hoop ("y") directions and
in-plane membrane shear and twist. Please see ITEM 299 in panda2.news.
NOTE: because of this capability there now exists a new input datum
in BEGIN.

36. PANDA2 now permits optimization of panels with Z-shaped stringers
and rings. Please see ITEM 375 in panda2.news. Part of ITEM 375 is
repeated here for convenience:

A new stiffener cross section was introduced into PANDA2: Z-shaped
stiffener for which the faying flange is considered to be riveted to the
panel skin at the midwidth of this flange. New sketches were introduced
into the PROMPT.DAT file and into SUBROUTINES PICTS, PICT2, and PICT3 of
the UTIL library, as follows: 

       Module with Z-shaped stiffener...

                                     |<-- w -->|
                                     ___________
                                     |   ^    .
                Segment No. 3 -----> |   |     .
                                     |   |      .Seg. No. 4
                    Seg. No. 2-.     |   h
                                .    |   |
              Seg. No. 1-.       .   |   |          .-Seg. No. 5
                        .      R  .  |   |         .(same as Seg.1)
                       .  _____I_____|   V        .
      ------------------------ V ------------------------
                          |    E     |
                          |    T     |
                          |<---b2--->|
      |<---- Module width  =  stiffener spacing, b ---->|


         Fig. 1 Sketch of the Z-shaped stiffener in the PROMPT.DAT file


    EXPLODED VIEW, SHOWING LAYERS and (SEGMENT, NODE) NUMBERS

                       Layer No. 1___
                                     . 
                                      . 
             (Segment,Node)=(4,1)____________(4,11)
                                         . 
                              (3,11)      .__Layer No. j
                                !  
             Layer No. 1 -----> ! <----------- Layer No. k
                                !   
             Layer No. 1-.      !   
 Layer No. 1-.            .     !         .-Layer No. 1
            .              .    !          .
           .             R  . (3,1)         .
          .   (2,1)______I______(2,11)       .
 ----------------------  V  ------------------------
 (1,1)    .       (1,11) E (5,1)              .   (5,11)
           .             T                     .
            .                                   .
             Layer No. m                         Layer No. m


        Fig. 2  Sketch of Z-shaped stiffener added to SUB. PICTS


The riveting of the faying flange to the panel skin is simulated by
a junction constraint on displacements u,v,w, and rotation ROT at the
midwidth of the faying flange (2,6) to the end node of Segment 1: (1,11).
The heel of the Z-shaped stringer (intersection of Segment 2 with Segment
3) is free to lift off the panel skin. This renders the behavior of
Z-shaped stringers considerably different from that of J-shaped stringers,
especially for non-optimum designs.


37. Alternative buckling solution installed in PANDA2 (July-Aug. 1998):
See ITEMs 438 and 444 in ..panda2/doc/panda2.news. A major new branch was
inserted into PANDA2: an alternate "PANDA-type" solution for buckling
("PANDA-type" in contrast to the discretized single module model). The
alternate solution is needed for squarish flat or curved panels subjected
to significant in-plane shear loading and flat or curved composite
laminates with significant anisotropic terms, B16, B26, or D16, D26, such
as occur especially in angle-ply laminates with few layers. In addition,
the alternate solution will handle buckling of stiffened panels in which
the stiffeners are treated as discrete rather than being "smeared"
(averaged) over the panel surface. The alternative solution involves
double trigonometric series expansions of the buckling modal displacement
components, u,v,w. The "alternative" solution is of the type described by
Timoshenko (pp 357-360, 1936 1st edition) by Whitney (pp 152-156), by
Jones, and by others. The alternative solution is now computed in the new
SUBROUTINE ALTSOL (BUCPAN2 library). Numerous comparisons have been made
with predictions from STAGS, and there is very good agreement. A new input
datum has been introduced into the *.OPT file (MAINSETUP/PANDAOPT input)
whereby the PANDA2 user chooses whether or not to use the alternate
solution branch. 

38. There is a new "skin"-ring discretized module in PANDA2. In this model
the stringers are smeared out, and the skin-stringer panel wall plus ring
cross section is segmented and discretized in a manner completely
analogous to the skin-stringer discretized module described in the
original long 1987 paper on PANDA2. For details, please read ITEM 463 in
the file ...panda2/doc/panda2.news. 

39. The model for prebuckling behavior of a ring and stringer stiffened
flat panel under normal pressure has been improved. Please read ITEM 470
in the file ...panda2/doc/panda2.news.

40. A processor called STAGSUNIT has been added. STAGSUNIT uses panda2
data plus some user input to generate the stags input files *.inp and
*.bin for a cylindrical panel or complete cylindrical shell stiffened by
rings and stringers with either rectangular, tee, jay, or z cross
sections. STAGSUNIT is analogous to STAGSMODEL. It builds up a stags model
with shell units rather than with a single element unit, as does
STAGSMODEL. (See "Recent PANDA2 references" No. [10] for more details about
STAGSUNIT).

41. With IQUICK = 0 models (discretized skin-stringer module model), the
user can now choose to retain the curvature of the panel skin in the
discretized module model. See ...panda2.news Item. No. 530 and Entries
807 and 809 in the prompting file, PROMPT.DAT.


42. September 2003 (defunct)
A new version of PANDA2 (called PANDA3) has been developed. In this version material
properties and most of the allowables can be decision variables in addition to the
panel cross section dimensions, layer thicknesses and winding angles. This work was
done at the request of Prof. Raphael Haftka of the Aerospace Dept. of the University
of Florida in Gainesville, FL.
In PANDA3 there are new input data required in BEGIN. For every variable PANDA3 asks, 
" Is this variable a decision variable candidate?" Therefore, input files for BEGIN
in PANDA2 (*.BEG) will not work on PANDA3 and input files for BEGIN in PANDA3 (*.BEG)
will not work on PANDA2.
The input files for DECIDE (*.DEC) and for MAINSETUP/PANDA2 (*.OPT) and for
CHOOSEPLOT (*.CPL) are the same for both PANDA2 and PANDA3.

43. January 2005
The capability of the computer program PANDA2 for the minimum-weight design
of stiffened panels and cylindrical shells is enhanced to permit the
adding of substiffeners with rectangular cross sections between adjacent major
stringers and rings. As a result many new buckling margins exist that
govern buckling over various domains and subdomains of the doubly stiffened
panel or shell. These generally influence the evolution of the design during
optimization cycles. The substiffeners may be stringers and/or rings or may
form an isogrid pattern. The effects of Local, inter-ring, and general buckling
modal imperfections are accounted for during optimization. Perfect and
imperfect cylindrical shells with external T-shaped stringers and T-shaped
rings and with and without external substringers and subrings and under combined
axial compression, external pressure, and in-plane shear are optimized
by multiple executions of a "global" optimizer called SUPEROPT. It is found
that from the point of view of minimum weight there is little advantage of
adding substiffeners. However, with substiffeners present the major stringers
and rings are spaced farther apart at the optimum than is the case with
no substiffeners. The weight of a cylindrical shell with substiffeners is much
less sensitive to the spacing of the major T-shaped stringers than is the case
for a cylindrical shell without substiffeners. The optimum designs obtained
by PANDA2 are evaluated by comparisons of buckling loads obtained from a
general-purpose finite element program called STAGS. Predictions from STAGS
agree well with those from PANDA2.

Please see the paper,  David Bushnell, OPTIMUM DESIGN OF STIFFENED PANELS WITH
SUBSTIFFENERS, AIAA PAPER 1932, 46th AIAA SDM Meeting, Austin, TX, April 2005
and ITEM NO. 600 in the file ...panda2/doc/panda2.news.

44. March 2007
A new section was added to the PANDA2 mainprocessor (SUBROUTINE STRUCT) in
which a knockdown factor is computed to compensate for the inherent
unconservativeness of smearing stringers. Details appear in Items 724 and
725 of ..panda2/doc/panda2.news.

44. March 2007
A new section was added to the PANDA2 mainprocessor (SUBROUTINE STRUCT) in
which a knockdown factor is computed to compensate for the inherent
unconservativeness of smearing stringers. Details appear in Items 724 and
725 of ..panda2/doc/panda2.news.

44. March 2007
A new section was added to the PANDA2 mainprocessor (SUBROUTINE STRUCT) in
which a knockdown factor is computed to compensate for the inherent
unconservativeness of smearing stringers. Details appear in Items 724 and
725 of ..panda2/doc/panda2.news.

44. March 2007
A new section was added to the PANDA2 mainprocessor (SUBROUTINE STRUCT) in
which a knockdown factor is computed to compensate for the inherent
unconservativeness of smearing stringers. Details appear in Items 724 and
725 of ..panda2/doc/panda2.news.

44. March 2007
A new section was added to the PANDA2 mainprocessor (SUBROUTINE STRUCT) in
which a knockdown factor is computed to compensate for the inherent
unconservativeness of smearing stringers. Details appear in Items 724 and
725 of ..panda2/doc/panda2.news.

44. March 2007
A new section was added to the PANDA2 mainprocessor (SUBROUTINE STRUCT) in
which a knockdown factor is computed to compensate for the inherent
unconservativeness of smearing stringers. Details appear in Items 724 and
725 of ..panda2/doc/panda2.news.

44. March 2007
A new section was added to the PANDA2 mainprocessor (SUBROUTINE STRUCT) in
which a knockdown factor is computed to compensate for the inherent
unconservativeness of smearing stringers. Details appear in Items 724 and
725 of ..panda2/doc/panda2.news.

44. March 2007
A new section was added to the PANDA2 mainprocessor (SUBROUTINE STRUCT) in
which a knockdown factor is computed to compensate for the inherent
unconservativeness of smearing stringers. Details appear in Items 724 and
725 of ..panda2/doc/panda2.news.

44. March 2007
A new section was added to the PANDA2 mainprocessor (SUBROUTINE STRUCT) in
which a knockdown factor is computed to compensate for the inherent
unconservativeness of smearing stringers. Details appear in Items 724 and
725 of ..panda2/doc/panda2.news.

44. March 2007
A new section was added to the PANDA2 mainprocessor (SUBROUTINE STRUCT) in
which a knockdown factor is computed to compensate for the inherent
unconservativeness of smearing stringers. Details appear in Items 724 and
725 of ..panda2/doc/panda2.news.

44. March 2007
A new section was added to the PANDA2 mainprocessor (SUBROUTINE STRUCT) in
which a knockdown factor is computed to compensate for the inherent
unconservativeness of smearing stringers. Details appear in Items 724 and
725 of ..panda2/doc/panda2.news.

44. March 2007
A new section was added to the PANDA2 mainprocessor (SUBROUTINE STRUCT) in
which a knockdown factor is computed to compensate for the inherent
unconservativeness of smearing stringers. Details appear in Items 724 and
725 of ..panda2/doc/panda2.news.

44. March 2007
A new section was added to the PANDA2 mainprocessor (SUBROUTINE STRUCT) in
which a knockdown factor is computed to compensate for the inherent
unconservativeness of smearing stringers. Details appear in Items 724 and
725 of ..panda2/doc/panda2.news.

44. March 2007
A new section was added to the PANDA2 mainprocessor (SUBROUTINE STRUCT) in
which a knockdown factor is computed to compensate for the inherent
unconservativeness of smearing stringers. Details appear in Items 724 and
725 of ..panda2/doc/panda2.news.

44. March 2007
A new section was added to the PANDA2 mainprocessor (SUBROUTINE STRUCT) in
which a knockdown factor is computed to compensate for the inherent
unconservativeness of smearing stringers. Details appear in Items 724 and
725 of ..panda2/doc/panda2.news.

44. March 2007
A new section was added to the PANDA2 mainprocessor (SUBROUTINE STRUCT) in
which a knockdown factor is computed to compensate for the inherent
unconservativeness of smearing stringers. Details appear in Items 724 and
725 of ..panda2/doc/panda2.news.

44. March 2007
A new section was added to the PANDA2 mainprocessor (SUBROUTINE STRUCT) in
which a knockdown factor is computed to compensate for the inherent
unconservativeness of smearing stringers. Details appear in Items 724 and
725 of ..panda2/doc/panda2.news.

44. March 2007
A new section was added to the PANDA2 mainprocessor (SUBROUTINE STRUCT) in
which a knockdown factor is computed to compensate for the inherent
unconservativeness of smearing stringers. Details appear in Items 724 and
725 of ..panda2/doc/panda2.news.

44. March 2007
A new section was added to the PANDA2 mainprocessor (SUBROUTINE STRUCT) in
which a knockdown factor is computed to compensate for the inherent
unconservativeness of smearing stringers. Details appear in Items 724 and
725 of ..panda2/doc/panda2.news.

44. March 2007
A new section was added to the PANDA2 mainprocessor (SUBROUTINE STRUCT) in
which a knockdown factor is computed to compensate for the inherent
unconservativeness of smearing stringers. Details appear in Items 724 and
725 of ..panda2/doc/panda2.news.

44. March 2007
A new section was added to the PANDA2 mainprocessor (SUBROUTINE STRUCT) in
which a knockdown factor is computed to compensate for the inherent
unconservativeness of smearing stringers. Details appear in Items 724 and
725 of ..panda2/doc/panda2.news.

44. March 2007
A new section was added to the PANDA2 mainprocessor (SUBROUTINE STRUCT) in
which a knockdown factor is computed to compensate for the inherent
unconservativeness of smearing stringers. Details appear in Items 724 and
725 of ..panda2/doc/panda2.news.

44. March 2007
A new section was added to the PANDA2 mainprocessor (SUBROUTINE STRUCT) in
which a knockdown factor is computed to compensate for the inherent
unconservativeness of smearing stringers. Details appear in Items 724 and
725 of ..panda2/doc/panda2.news.

44. March 2007
A new section was added to the PANDA2 mainprocessor (SUBROUTINE STRUCT) in
which a knockdown factor is computed to compensate for the inherent
unconservativeness of smearing stringers. Details appear in Items 724 and
725 of ..panda2/doc/panda2.news.

44. March 2007
A new section was added to the PANDA2 mainprocessor (SUBROUTINE STRUCT) in
which a knockdown factor is computed to compensate for the inherent
unconservativeness of smearing stringers. Details appear in Items 724 and
725 of ..panda2/doc/panda2.news.

44. March 2007
A new section was added to the PANDA2 mainprocessor (SUBROUTINE STRUCT) in
which a knockdown factor is computed to compensate for the inherent
unconservativeness of smearing stringers. Details appear in Items 724 and
725 of ..panda2/doc/panda2.news.

44. March 2007
A new section was added to the PANDA2 mainprocessor (SUBROUTINE STRUCT) in
which a knockdown factor is computed to compensate for the inherent
unconservativeness of smearing stringers. Details appear in Items 724 and
725 of ..panda2/doc/panda2.news.

44. March 2007
A new section was added to the PANDA2 mainprocessor (SUBROUTINE STRUCT) in
which a knockdown factor is computed to compensate for the inherent
unconservativeness of smearing stringers. Details appear in Items 724 and
725 of ..panda2/doc/panda2.news.

44. March 2007
A new section was added to the PANDA2 mainprocessor (SUBROUTINE STRUCT) in
which a knockdown factor is computed to compensate for the inherent
unconservativeness of smearing stringers. Details appear in Items 724 and
725 of ..panda2/doc/panda2.news.

44. March 2007
A new section was added to the PANDA2 mainprocessor (SUBROUTINE STRUCT) in
which a knockdown factor is computed to compensate for the inherent
unconservativeness of smearing stringers. Details appear in Items 724 and
725 of ..panda2/doc/panda2.news.

44. March 2007
A new section was added to the PANDA2 mainprocessor (SUBROUTINE STRUCT) in
which a knockdown factor is computed to compensate for the inherent
unconservativeness of smearing stringers. Details appear in Items 724 and
725 of ..panda2/doc/panda2.news.

44. March 2007
A new section was added to the PANDA2 mainprocessor (SUBROUTINE STRUCT) in
which a knockdown factor is computed to compensate for the inherent
unconservativeness of smearing stringers. Details appear in Items 724 and
725 of ..panda2/doc/panda2.news.

44. March 2007
A new section was added to the PANDA2 mainprocessor (SUBROUTINE STRUCT) in
which a knockdown factor is computed to compensate for the inherent
unconservativeness of smearing stringers. Details appear in Items 724 and
725 of ..panda2/doc/panda2.news.

44. March 2007
A new section was added to the PANDA2 mainprocessor (SUBROUTINE STRUCT) in
which a knockdown factor is computed to compensate for the inherent
unconservativeness of smearing stringers. Details appear in Items 724 and
725 of ..panda2/doc/panda2.news.

44. March 2007
A new section was added to the PANDA2 mainprocessor (SUBROUTINE STRUCT) in
which a knockdown factor is computed to compensate for the inherent
unconservativeness of smearing stringers. Details appear in Items 724 and
725 of ..panda2/doc/panda2.news.

44. March 2007
A new section was added to the PANDA2 mainprocessor (SUBROUTINE STRUCT) in
which a knockdown factor is computed to compensate for the inherent
unconservativeness of smearing stringers. Details appear in Items 724 and
725 of ..panda2/doc/panda2.news.

44. March 2007
A new section was added to the PANDA2 mainprocessor (SUBROUTINE STRUCT) in
which a knockdown factor is computed to compensate for the inherent
unconservativeness of smearing stringers. Details appear in Items 724 and
725 of ..panda2/doc/panda2.news.

44. March 2007
A new section was added to the PANDA2 mainprocessor (SUBROUTINE STRUCT) in
which a knockdown factor is computed to compensate for the inherent
unconservativeness of smearing stringers. Details appear in Items 724 and
725 of ..panda2/doc/panda2.news.

44. March 2007
A new section was added to the PANDA2 mainprocessor (SUBROUTINE STRUCT) in
which a knockdown factor is computed to compensate for the inherent
unconservativeness of smearing stringers. Details appear in Items 724 and
725 of ..panda2/doc/panda2.news.

44. March 2007
A new section was added to the PANDA2 mainprocessor (SUBROUTINE STRUCT) in
which a knockdown factor is computed to compensate for the inherent
unconservativeness of smearing stringers. Details appear in Items 724 and
725 of ..panda2/doc/panda2.news.

44. March 2007
A new section was added to the PANDA2 mainprocessor (SUBROUTINE STRUCT) in
which a knockdown factor is computed to compensate for the inherent
unconservativeness of smearing stringers. Details appear in Items 724 and
725 of ..panda2/doc/panda2.news.

44. March 2007
A new section was added to the PANDA2 mainprocessor (SUBROUTINE STRUCT) in
which a knockdown factor is computed to compensate for the inherent
unconservativeness of smearing stringers. Details appear in Items 724 and
725 of ..panda2/doc/panda2.news.

44. March 2007
A new section was added to the PANDA2 mainprocessor (SUBROUTINE STRUCT) in
which a knockdown factor is computed to compensate for the inherent
unconservativeness of smearing stringers. Details appear in Items 724 and
725 of ..panda2/doc/panda2.news.

44. March 2007
A new section was added to the PANDA2 mainprocessor (SUBROUTINE STRUCT) in
which a knockdown factor is computed to compensate for the inherent
unconservativeness of smearing stringers. Details appear in Items 724 and
725 of ..panda2/doc/panda2.news.


                         CAPABILITY OF STAGS

STAGS (STructural Analysis of General Shells) is a finite element code for
general-purpose nonlinear analysis of stiffened shell structures of
arbitrary shape and complexity. Its capabilities include stress,
stability, vibration, and transient analyses with both material and
geometric nonlinearities permitted in all analysis types. Currently a new
version of STAGS, scheduled for release through COSMIC in 1992, is under
development. New enhancements include a higher order thick shell element,
more advanced nonlinear solution strategies, and more comprehensive
post-processing features such as a link with PATRAN [45]. 

Research and development of STAGS by Brogan, Almroth, Rankin, Stanley,
Cabiness, Stehlin and others of the Computational Mechanics Department of
the Lockheed Palo Alto Research Laboratory has been under continuous
sponsorship from U.S. government agencies and internal Lockheed funding
for the past 20 years.  During this time particular emphasis has been
placed on improvement of the capability to solve difficult nonlinear
problems such as the prediction of the behavior of axially compressed
stiffened panels loaded far into their locally postbuckled states. STAGS
has been extensively used worldwide for the evaluation of stiffened panels
and shells loaded well into their locally postbuckled states. See [7], for
example. 

A large rotation algorithm that is independent of the finite element
library has been incorporated into STAGS [46].  With this algorithm there
is no artificial stiffening due to large rotations.  The finite elements
in the STAGS library do not store energy under arbitrary rigid-body motion
and the first and second variations of the strain energy are consistent.
These properties lead to quadratic convergence during Newton iterations. 

Solution control in nonlinear problems includes specification of load
levels or use of the advanced Riks-Crisfield path parameter that enables
traversal of limit points into the post-buckling regime. Two load systems
with different histories (Load Sets A and B) can be defined and controlled
separately during the solution process. Flexible restart procedures permit
switching from one strategy to another during an analysis.  This includes
shifts from bifurcation buckling to nonlinear collapse analyses and back
and shifts from static to transient and transient to static analyses with
modified boundary conditions and loading.  STAGS provides solutions to the
generalized eigenvalue problem for buckling and vibration from a linear or
nonlinear stress state. 

Quadric surfaces can be modeled with minimal user input as individual
substructures called "shell units" in which the analytic geometry is
represented exactly. "Shell units" can be connected along edges or
internal grid lines with partial or complete compatibility.  In this way
complex structures can be assembled from relatively simple units.
Alternatively, a structure of arbitrary shape can be modelled with use of
an "element unit". 

Geometric imperfections can be generated automatically in a variety of
ways, thereby permitting imperfection-sensitivity studies to be performed.
 For example,  imperfections can be generated by superposition of several
buckling modes determined from previous STAGS analyses of a given case. 

A variety of material models is available, including both plasticity and
creep.  STAGS handles isotropic and anisotropic materials, including
composites consisting of up to 60 layers of arbitrary orientation. Four
plasticity models are available, including isotropic strain hardening, the
White Besseling (mechanical sublayer model), kinematic strain hardening,
and deformation theory. 

Two independent load sets, each composed from simple parts that may be
specified with minimal input, define a spatial variation of loading. Any
number of point loads, prescribed displacements, line loads, surface
tractions, thermal loads, and "live" pressure (hydrostatic pressure which
remains normal to the shell surface throughout large deformations) can be
combined to make a load set. For transient analysis the user may select
from a menu of loading histories, or a general temporal variation may be
specified in a user-written subroutine. 

Boundary conditions (B.C.) may be imposed either by reference to certain
standard conditions or by the use of single- and multi-point constraints.
Simple support, symmetry, antisymmetry, clamped, or user-defined B.C. can
be defined on a "shell unit" edge.  Single-point constraints which allow
individual freedoms to be free, fixed, or a prescribed non-zero value may
be applied to grid lines and surfaces in "shell units" or "element units".
 A useful feature for buckling analysis allows these constraints to differ
for the prestress and eigenvalue analyses. Langrangian constraint
equations containing up to 100 terms may be defined to impose multi-point
constraints. 

STAGS has a variety of finite elements suitable for the analysis of
stiffened plates and shells.  Simple four node quadrilateral plate
elements with a cubic lateral displacement field (called "410" and "411"
elements) are effective and efficient for the prediction of postbuckling
thin shell response.  A linear (410) or quadratic (411) membrane
interpolation can be selected.  For thicker shells in which transverse
shear deformation is important, STAGS provides the Assumed Natural Strain
(ANS) nine node element (called "480" element).  A two node beam element
compatible with the four node quadrilateral plate element is provided to
simulate stiffeners and beam assemblies.  Other finite elements included
in STAGS are described in the STAGS literature [47]. 


              PANDA2 - to - STAGS  MODEL GENERATION

A PANDA2 processor, executed by the command "STAGSMODEL", creates
input files for STAGS corresponding to panel configurations optimized with
PANDA2.  With STAGSMODEL/STAGS the load-carrying capacity of optimum
designs obtained by PANDA2 can be checked without the user having to spend
time setting up elaborate finite element models for STAGS from directions
in the STAGS user's manual. The STAGSMODEL processor can be used to create
a succession of STAGS models by means of which bifurcation buckling
behavior and nonlinear post-local-buckling behavior of a panel optimized
by PANDA2 can be determined. This is a valuable feature because panels
designed by the approximate methods used in PANDA2 can now easily be
"tested" by a widely used, rigorous, general, nonlinear finite element
general purpose program. 

The STAGSMODEL processor creates a finite element model of the part of a
panel between adjacent rings. The stiffness along the axes of the rings is
accounted for in the finite element model created by STAGSMODEL/STAGS. The
panel can be loaded by any combination of uniform axial load Nx, uniform
hoop load, Ny, uniform in-plane shear Nxy, and uniform normal pressure p. 

LIMITATIONS IN STAGSMODEL: Thermal loading and edge moments cannot yet be
included. Truss-core sandwich panels cannot be handled. Also, if the panel
is axially stiffened the axially loaded edges cannot rotate in either the
prebuckling or bifurcation buckling phases of the analysis. The material
must remain elastic. The loading/end shortening must be uniform. 
STAGSMODEL will not handle truss-core stiffened panels or
isogrid-stiffened panels. 

Depending on a user-selected index, the normal projection of the two edges
parallel to the stringers may be forced to remain straight or may be
allowed to undergo in-plane warping (varying circumferential
displacement). At the axially loaded ends of the panel the cross sections
of the stringers are not allowed to warp. In non-bifurcation phases of
analyses the axial displacement u is zero at one end of the STAGS model
and is uniform at the other end. The user chooses whether or not stringer
sidesway is permitted at the axially loaded ends of the panel. 

The entire STAGS model is created in what in STAGS jargon is called an
"element unit".  The stringer web(s) and outstanding flange are also
modelled with finite elements. The line loads Nx, Ny and Nxy and pressure
p are modelled as nodal point loads.  The user may choose to employ 410,
411, or 480 elements [47]. If the user chooses 410 elements all segments
of the panel are modelled with 410 elements except the bases under the
stringers, which are modelled with 411 elements. If the user chooses
either 411 or 480 elements the entire panel is modelled with those
elements. 

 *************** NOTE **** STAGSUNIT **** NOTE *********************
ANOTHER NEW PANDA2-to-STAGS processor called STAGSUNIT: see No. 40
above and "recent PANDA2 reference" No. [10] listed below.
 ******************************************************************


                       REFERENCES

[1] A. W. Leissa, ``Buckling of laminated composite plates and shell
panels," AFWAL-TR-85-3069, Air Force Wright Aeronautical
Laboratories, Wright-Patterson AFB, Ohio 45433, June, 1985. 

[2] J. F. M. Wiggenraad, ``Postbuckling of thin-walled composite
structures - design, analysis, and experimental verification,"
National Aerospace Laboratory (NLR), The Netherlands, Memorandum
SC-86-013 U, Feb. 28, 1986. 

[3] R. R. Arnold and J. C. Parekh, ``Buckling, postbuckling, and
failure of flat and shallow-curved, edge-stiffened composite plates
subject to combined axial compression and shear loads", Presented at
27th SDM Meeting, San Antonio, Tx., April 1986, AIAA Paper No.
86-1027-CP, 1986, Proceedings pp. 769-782. 

[4] D. Bushnell, ``PANDA-Interactive program for minimum weight
design of stiffened cylindrical panels and shells,'' Computers and
Structures, 16, 1983, pp. 167-185.

[4b] D. Bushnell, ``Theoretical basis of the PANDA computer program
for preliminary design of stiffened panels under combined in-plane
loads," Computers and Structures, Vol. 27, No. 4, pp 541-563 (1987) 

[5] D. Bushnell,  Computerized Buckling Analysis of Shells, Martinus
Nijhoff Publishers, The Netherlands, 1985; reprinted 1989 by Kluwer.

[6] M. Baruch and J. Singer, ``Effect of eccentricity of stiffeners
on the general instability of stiffened cylindrical shells under
hydrostatic pressure," Journal of Mechanical Engineering Science, 5,
(1) (1963) pp.23-27. 

[7] J. H. Starnes,Jr., N. F. Knight,Jr. and M. Rouse, ``Postbuckling
behavior of selected flat stiffened graphite- epoxy panels loaded in
compression,'' AIAA Paper 82-0777, presented at AIAA 23rd Structures,
Structural Dynamics, and Materials Conference, New Orleans, May,
1982.  See also, AIAA J., 23, (8) (1985) pp.1236-1246. 

[8] E. E. Spier, ``On experimental versus theoretical incipient
buckling of narrow graphite/epoxy plates in compression," Proc. AIAA
21st SDM Conference, AIAA Paper 80-0686-CP, May, 1980. 

[9] E. E. Spier, ``Local buckling, postbuckling, and crippling
behavior of graphite-epoxy short thin-walled compression members,"
Naval Air Systems Command, Washington, D. C., NASC-N00019-80-C-0174,
July 1981. 

[10] M. P. Renieri and R. A. Garrett, ``Investigation of the local
buckling, postbuckling and crippling behavior of graphite/epoxy short
thin-walled compression members," McDonnell Douglas Corporation, St.
Louis, Missouri, MDC A7091, July 1981. 

[11] M. P. Renieri and R. A. Garrett, ``Postbuckling fatigue behavior
of flat stiffened graphite/epoxy panels under shear loading," Naval
Air Development Center, Warminster, PA, NADC-81-168-60, July 1982. 

[12] L. W. Rehfield and A. D. Reddy, ``Observations on compressive
local buckling, postbuckling, and crippling of graphite/epoxy
airframe structure," Proc. 27th AIAA SDM Conference, AIAA Paper
86-0923-CP, May 1986. 

[13] T. Weller, G. Messer and A. Libai, ``Repeated buckling of
graphite epoxy shear panels with bonded metal stiffeners," Dept. of
Aeronautical Engineering, Technion, Haifa, Israel, TAE No. 546,
August 1984. 

[14] B. L. Agarwal, ``Postbuckling behavior of composite, stiffened,
curved panels loaded in compression," Experimental Mechanics, Vol.
22, June 1982. 

[15] C. Blaas and J. F. M. Wiggenraad, ``Development and test
verification of the ARIANE 4 interstage 2/3 in CFRP", Proceedings of
the AIAA/ASME 27th Structures, Structural Dynamics and Materials
Conference, Part 1, May 1986, pp. 307-313. 

[16] J. M. T. Thompson, J. D. Tulk and A. C. Walker, ``An
experimental study of imperfection-sensitivity in the interactive
buckling of stiffened plates", in:  Buckling of Structures, B.
Budiansky, ed., Springer-Verlag, 1976, pp 149-159. 

[17] Bushnell, D., Holmes, A.M.C, Flaggs, D.L., and McCormick, P.J.,
``Optimum design, fabrication, and test of graphite-epoxy, curved,
stiffened, locally buckled panels loaded in axial compression,'',
in BUCKLING OF STRUCTURES, edited by I. Elishakoff, et al, Elsevier
Science Publishers, Amsterdam, pp 61-131 (1988)

[18] N. R. Bauld, Jr. and N. S. Khot, ``A numerical and experimental
investigation of the buckling behavior of composite panels",
Computers and Structures, 15 (1982) pp. 393-403. 

[19] N. S. Khot and N. R. Bauld, Jr., ``Further comparison of the
numerical and experimental buckling behaviors of composite panels,"
Computers and Structures, 17, (1983) pp. 61-68. 

[20] Y. Zhang and F. L. Matthews, ``Postbuckling behavior of
anisotropic laminated plates under pure shear and shear combined with
compressive loading", AIAA Journal, 22, (2), (1984) pp 281-286. 

[21] B. O. Almroth and F. A. Brogan, ``The STAGS Computer Code", NASA
CR-2950, Nasa Langley Research Center, Hampton, Va., 1978. 

[22] C. C. Rankin, P. Stehlin and F. A. Brogan, ``Enhancements to the
STAGS computer code", NASA CR 4000, NASA Langley Research Center,
Hampton, Va, November 1986. See also, G. A. Thurston, F. A. Brogan
and P. Stehlin, ``Postbuckling analysis using a general purpose
code", AIAA Journal, 24, (6) (1986) pp. 1013-1020. 

[23] Graves-Smith, T.R. and Sridharan, S., ``A finite strip method
for the post-locally-buckled analysis of plate structures,'' Int.
J. Mech. Sci., Vol. 20, pp 833-843 (1978)

[24] Stoll, F. and Gurdal, Z., ``Nonlinear analysis of compressively
loaded linked-plate structures,'' AIAA Paper 90-0968-CP,
Proceedings 31st AIAA/ASME Structures, Structural Dynamics,
and Materials Meeting, pp 903-913 (1990).

[24b] Stoll, F. and Gurdal, Z., and Starnes, J. H., Jr., ``A method for
the geometrically nonlinear analysis of compressively loaded prismatic
composite structures,'' VIPSU Center for Composite Materials and
Structures Report CCMS-91-03 (VPI-E-91-01), February, 1991

[24c] Shin, D. K., Gurdal, Z., and Griffin, O. H., Jr., ``Minimum
weight design of laminated composite plates for postbuckling
performance,'' AIAA Paper 91-0969-CP, Proceedings 32nd AIAA/ASME
Structures, Structural Dynamics, and Materials Meeting, pp 257-266 (1991)

[25] Riks, E., ``A finite strip method for the buckling and
postbuckling analysis of stiffened panels in wing box structures,''
National Aerospace Laboratory (NLR), Report NLR CR 89383 L, The
Netherlands, November 1989

[26] M. S. Anderson and W. J. Stroud, ``General panel sizing computer
code and its application to composite structural panels,'' AIAA
Journal, 17, (8) (1979) pp. 892-897. 

[27] W. J. Stroud and M. S. Anderson, ``PASCO: Structural panel
analysis and sizing code, capability and analytical foundations,''
NASA TM-80181, NASA Langley Research Center, Hampton, Va., 1981. 

[28] W. J. Stroud, W. H. Greene and M. S. Anderson, ``Buckling loads
of stiffened panels subjected to combined longitudinal compression
and shear: Results obtained with PASCO, EAL, and STAGS computer
programs,'' NASA TP 2215, Nasa Langley Research Center, Hampton, Va.,
January 1984. 

[29] J. N. Dickson, R. T. Cole and J. T. S. Wang, ``Design of
stiffened composite panels in the post-buckling range," in: Fibrous
Composites in Structural Design, E. M. Lenoe, D. W. Oplinger, and J.
J. Burke, editors, Plenum Press, New York, 1980, pp 313-327. 

[30] J. N. Dickson, S. B. Biggers, and J. T. S. Wang, ``Preliminary
design procedure for composite panels with open-section stiffeners
loaded in the post-buckling range," in:  Advances in Composite
Materials, A. R. Bunsell, et al,editors, Pergamon Press Ltd., Oxford,
England, 1980, pp 812-825. 

[31] J. N. Dickson and S. B. Biggers, ``POSTOP: Postbuckled open-
stiffened optimum panels, theory and capability", NASA Langley
Research Center, Hampton, Va., NASA Contractor Report from NASA
Contract NAS1 - 15949, May 1982. 

[32] Butler, R. and Williams, F. W., ``Optimuam design features
of VICONOPT, an exact buckling program for prismatic assemblies of
anisotropic plates,''  AIAA Paper 90-1068-CP, 
Proceedings 31st AIAA/ASME Structures, Structural Dynamics,
and Materials Meeting, pp 1289-1299

[33] Williams, F. W., Kennedy, D., Anderson, M.S., ``Analysis features
of VICONOPT, an exact buckling and vibration program for prismatic
assemblies of anisotropic plates,'', AIAA Paper 90-0970-CP,
Proceedings 31st AIAA/ASME Structures, Structural Dynamics,
and Materials Meeting, pp 920-929

[34] Peng, M-H and Sridharan, S., ``Optimized design of stiffened
panels subject to interactive buckling,'' AIAA Paper 90-1067-CP,
Proceedings 31st AIAA/ASME Structures, Structural Dynamics, and
Materials Meeting, pp 1279-1288 

[35] D. Bushnell, ``PANDA2-Program for minimum weight design of
stiffened, composite, locally buckled panels", Computers and
Structures, Vol. 25 (1987) pp. 469-605. See also, D. Bushnell,
``Optimization of composite, stiffened, imperfect panels under
combined loads for service in the postbuckling regime", Computer
Methods in Applied Mechanics and Engineering, Vol. 103, pp. 43-114,
(1993) (volume in honor of Hans Besseling's 65th birthday).

[36] G. N. Vanderplaats, ``CONMIN-a FORTRAN program for constrained
function minimization,'' NASA TM X 62-282, version updated in March,
1975, Ames Research Center, Moffett Field, CA (Aug.1973) See also, G.
N. Vanderplaats and F. Moses, ``Structural optimization by methods of
feasible directions,'' Computers and Structures, 3 (1973) pp.
739-755. 

[37] Vanderplaats, G. N., "ADS--a FORTRAN program for automated
design synthesis, Version 2.01", Engineering Design Optimization,
Inc, Santa Barbara, CA, January, 1987 

[38] Vanderplaats, G. N. and Sugimoto, H., "A general-purpose
optimization program for engineering design", Computers and
Structures, Vol. 24, pp 13-21, 1986 
 
[39] D. Bushnell, ``BOSOR4: Program for stress, buckling, and
vibration of complex shells of revolution,'' Structural Mechanics
Software Series - Vol. 1, (N. Perrone and W. Pilkey, editors),
University Press of Virginia, Charlottesville, 1977, pp. 11-131. See
also Computers and Structures, Vol. 4, (1974) pp. 399-435;  AIAA J,
Vol. 9, No. 10, (1971) pp. 2004-2013; Structural Analysis Systems,
Vol. 2, A. Niku-Lari, editor, Pergamon Press, Oxford, 1986, pp.
25-54, and Computers and Structures, 18, (3), (1984) pp. 471-536. 
 
[40] W. T. Koiter, ``Het Schuifplooiveld by Grote Overshrijdingen van
de Knikspanning,'' Nationaal Luchtvaart Laboratorium, The
Netherlands, Report X295, November 1946 (in Dutch) 

[41] Bushnell, D., "Improvements to PANDA2", May 1989, revised November
1991, Volume 1 and Vol. 2, parts 1 and 2, unpublished "handout"
distributed with PANDA2
(These documents are no longer distributed with PANDA2 literature. Please
see "Recent PANDA2 references" listed below.) 

[42] Bushnell, D., a file ....panda2/doc/panda2.news, kept up-to-date.

[43] Bushnell, D., "Truss-core sandwich design via PANDA2", 
COMPUTERS AND STRUCTURES, Vol. 44, No. 5, pp 1091-1119 (1992)

[44] Roark, R. J. and Young, W. C., FORMULAS FOR STRESS AND STRAIN,
5th Edition, McGraw-Hill, 1975 (in particular Table 10, p. 158,
Formula 2d) 

[45] "PATRAN-Plus User Manual, Release 2.4, PDA Engineering, Costa Mesa,
California, September 1989

[46] Rankin, C. C. and Brogan, F. A., "An element independent corotational
procedure for the treatment of large rotations", Journal of Pressure
Vessel Technology, Vol. 108, pp 165-174 (1986)

[47] STAGS user's manual (latest version, on which the work reported here
is based, has not yet been released.)

[48] Brogan, F.A., Bushnell, W. D., Cabiness, H., and Rankin, C., STAGSPP,
a postprocessor for STAGS (not released as of this writing).


                       Recent PANDA2 references

[1] Bushnell, D. and Bushnell, W. D., "Minimum-weight design of a
stiffened panel via PANDA2 and evaluation of the optimized panel via
STAGS", Computers and Structures, Vol. 50, no. 4, p569-602 (1994)

[2] Bushnell, D. and Bushnell, W. D., "Optimum design of composite
stiffened panels under combined loading", Computers and Structures,
Vol. 55, No. 5, pp 819-856 (1995)

[3] Bushnell, D., "Optimization of composite, stiffened, imperfect
panels under combined loads for service in the postbuckling regime",
Computer Methods in Applied Mechanics and Engineering, Vol. 103, pp
43-114 (1993) 

[4] Bushnell, D. and Bushnell, W.,"Approximate method for the optimum
design of ring and stringer stiffened cylindrical panels and shells
with local, inter-ring, and general buckling modal imperfections", 
Computers and Structures, Vol. 59, No. 3, pp. 489-527 (1996)

[5] Bushnell, D., "Optimization of panels with riveted Z-shaped
          stiffeners via PANDA2, in Advances in the Mechanics of Plates and
          Shells", Durban, D, Givoli, D., and Simmonds, J.G., Eds, Kluwer
          Academic Publishers,  79-102, 2001. See also,  Proc. 39th AIAA
          Structures, Materials and Dynamics meeting, 2357-2388, AIAA. 1998. 

[6] Bushnell, D., "Optimum design via PANDA2 of composite sandwich
panels with honeycomb or foam cores,  Paper 97-1142, 38th AIAA Structures,
Structural Dynamics and Materials Conference, April, 1997.

[7] Bushnell, D., Rankin, C. C., Riks, E, "Optimization of stiffened
panels in which mode jumping is accounted for", AIAA Paper 97-1141, 38th
AIAA SDM Conference, April 1997 

[8] Bushnell, D., "Optimization of panels with riveted Z-shaped
stiffeners via PANDA2, in Advances in the Mechanics of Plates and
Shells", Durban, D, Givoli, D., and Simmonds, J.G., Eds, Kluwer
Academic Publishers,  79-102, 2001. See also,  Proc. 39th AIAA
Structures, Materials and Dynamics meeting, 2357-2388, AIAA. 1998. 

[9] Bushnell, D., Jiang, H., and Knight, N. F., Jr., Additional buckling
solutions in PANDA2", 40th AIAA SDM Conference, pp 302-345, Paper 99-1233,
April 1999

[10] David Bushnell and Charles C. Rankin, Optimization of perfect and
imperfect ring and stringer stiffened cylindrical shells with PANDA2 and
evaluation of optimum designs with STAGS, AIAA 43rd SDM Meeting, April
2002, AIAA Paper 2002-1408

[11] David Bushnell, Global optimum design of externally pressurized
isogrid stiffened cylindrical shells with added T-rings, Int. J. of
Non-Linear Mechanics, Vol. 37, Nos 4-5, (special "Arbocz" issue)
pp801-831 (2002)

[12] David Bushnell, Optimum design of stiffened panels with substiffeners,
AIAA Paper 1932, 46th AIAA SDM Meeting, Austin, TX, April 2005

[13] Bushnell, D. and Rankin, C.C.,
"Difficulties in optimization of imperfect stiffened
cylindrical shells",
AIAA Paper 1943, 47th AIAA Structures, Structural Dynamics and
Materials Meeting, Newport RI, April 2006

[14] Bushnell, D.
"Optimization of an axially compressed ring and stringer stiffened
cylindrical shell with a general buckling modal imperfection",
AIAA Paper 2007-2216, 48th AIAA SDM Meeting, Honolulu, Hawaii,
April 2007.

